\href{https://travis-ci.org/Erriez/ErriezRobotDyn4DigitDisplay}{\tt }

This is a Robot\+Dyn 4-\/digit 7-\/segment L\+ED display library for Arduino. The P\+CB contains a two wire \href{https://github.com/Erriez/ErriezTM1637}{\tt T\+M1637 L\+ED / button} controller.



{\bfseries Note}\+: This library uses the double-\/dot to display a time. The L\+ED dots per segment are not wired and cannot be controlled.

\subsection*{Library features}


\begin{DoxyItemize}
\item Set brightness (0..7)
\item Set digit (0..3)
\item Control all individual segments per digit
\item Control double dots (on/off)
\item Display time (hours\+:minutes)
\item Display decimal value (-\/999..9999) with optional padding
\item Display hexadecimal value (0...0x\+F\+F\+FF) with optional padding
\end{DoxyItemize}

\subsection*{Hardware}

{\bfseries Connection display with Arduino}

\tabulinesep=1mm
\begin{longtabu} spread 0pt [c]{*2{|X[-1]}|}
\hline
\rowcolor{\tableheadbgcolor}\PBS\centering {\bf Display }&{\bf Arduino U\+NO / Nano / Pro Mini / Leonardo / Mega2560 / E\+S\+P8266 / Lolin32  }\\\cline{1-2}
\endfirsthead
\hline
\endfoot
\hline
\rowcolor{\tableheadbgcolor}\PBS\centering {\bf Display }&{\bf Arduino U\+NO / Nano / Pro Mini / Leonardo / Mega2560 / E\+S\+P8266 / Lolin32  }\\\cline{1-2}
\endhead
\PBS\centering G\+ND &G\+ND \\\cline{1-2}
\PBS\centering V\+CC &5V (or 3.\+3V) \\\cline{1-2}
\PBS\centering C\+LK &Any D\+I\+G\+I\+T\+AL pin \\\cline{1-2}
\PBS\centering D\+IO &Any D\+I\+G\+I\+T\+AL pin \\\cline{1-2}
\end{longtabu}
Other M\+CU\textquotesingle{}s may work, but are not tested.

\subsection*{Examples}

Examples $\vert$ Erriez Robot\+Dyn 4-\/digit display $\vert$ \href{https://github.com/Erriez/ErriezRobotDyn4DigitDisplay/blob/master/examples/7SegementDisplayDemo/7SegementDisplayDemo.ino}{\tt 7\+Segement\+Display\+Demo}

\subsection*{Documentation}


\begin{DoxyItemize}
\item \href{https://erriez.github.io/ErriezRobotDyn4DigitDisplay}{\tt Online H\+T\+ML}
\item \href{https://github.com/Erriez/ErriezRobotDyn4DigitDisplay/raw/gh-pages/latex/ErriezRobotDyn4DigitDisplay.pdf}{\tt Download P\+DF}
\end{DoxyItemize}

\subsection*{Usage}

{\bfseries Initialization}


\begin{DoxyCode}
1 \{c++\}
2 #include <RobotDyn4DigitDisplay.h>
3 
4 // Connect display pins to the Arduino DIGITAL pins
5 #if defined(ARDUINO\_ARCH\_AVR)
6 #define TM1637\_CLK\_PIN      2
7 #define TM1637\_DIO\_PIN      3
8 #elif defined(ARDUINO\_ESP8266\_WEMOS\_D1MINI) || defined(ESP8266\_WEMOS\_D1MINI) ||
       defined(ARDUINO\_ESP8266\_NODEMCU)
9 #define TM1637\_CLK\_PIN      D2
10 #define TM1637\_DIO\_PIN      D3
11 #elif defined(ARDUINO\_LOLIN32)
12 #define TM1637\_CLK\_PIN      0
13 #define TM1637\_DIO\_PIN      4
14 #else
15 #error "May work, but not tested on this target"
16 #endif
17 
18 // Create display object
19 RobotDyn4DigitDisplay display(TM1637\_CLK\_PIN, TM1637\_DIO\_PIN);
20 
21 void setup()
22 \{
23     // Initialize TM1637
24     display.begin();
25 \}
\end{DoxyCode}


{\bfseries Clear display}


\begin{DoxyCode}
1 \{c++\}
2 // Clear display
3 display.clear();    // \_ \_ \_ \_
\end{DoxyCode}


{\bfseries Set brightness}


\begin{DoxyCode}
1 \{c++\}
2 // Set brightness
3 display.setBrightness(0); // Minimum
4 display.setBrightness(7); // Maximum
\end{DoxyCode}


{\bfseries Display time}


\begin{DoxyCode}
1 \{c++\}
2 // Display time
3 display.time(11, 59);   // 1 1 : 5 9
\end{DoxyCode}


{\bfseries Control time double dot}


\begin{DoxyCode}
1 \{c++\}
2 display.doubleDots(true);   // Turn double dot on
3 display.doubleDots(false);  // Turn double dot off
\end{DoxyCode}


{\bfseries Display decimal value}


\begin{DoxyCode}
1 \{c++\}
2 // Display decimal values
3 display.dec(-999);  // - 9 9 9
4 display.dec(-1);    // \_ \_ - 1
5 display.dec(0);     // \_ \_ \_ 0
6 display.dec(1);     // \_ \_ \_ 1
7 display.dec(123);   // \_ 1 2 3
8 display.dec(9999);  // 9 9 9 9
9 display.dec(10000); // - - - -
10 
11 // Display decimal values with padding
12 display.dec(1);     // \_ \_ \_ 1  (Default no padding)
13 display.dec(1, 2);  // \_ \_ 0 1  (2 digits padding)
14 display.dec(1, 3);  // \_ 0 0 1  (3 digits padding)
15 display.dec(1, 4);  // 0 0 0 1  (4 digits padding)
16 
17 display.dec(34, 3); // \_ 0 3 4  (2 digits padding)
\end{DoxyCode}


{\bfseries Display hexadecimal value}


\begin{DoxyCode}
1 \{c++\}
2 // Display hexadecimal values
3 display.dec(0x0000);    // 0 0 0 0
4 display.dec(0x1234);    // 1 2 3 4
5 display.dec(0xABCD);    // A B C D
6 display.dec(0xBEEF);    // B E E F
7 
8 // Display hexadecimal values with padding
9 display.hex(0x0001);    // \_ \_ \_ 1  (Default no padding)
10 display.hex(0x0001, 2); // \_ \_ 0 1  (2 digits padding)
11 display.hex(0x0001, 3); // \_ 0 0 1  (3 digits padding)
12 display.hex(0x0001, 4); // 0 0 0 1  (4 digits padding)
13 
14 display.hex(0x0034, 3); // \_ 0 3 4  (2 digits padding)
\end{DoxyCode}


{\bfseries Control individual digits}


\begin{DoxyCode}
1 \{c++\}
2 // Display individual digits: 1 2 3 4
3 display.digit(0, 1);
4 display.digit(1, 2);
5 display.digit(2, 3);
6 display.digit(3, 4);
\end{DoxyCode}


{\bfseries Special characters}


\begin{DoxyCode}
1 \{c++\}
2 Control individual LED-segments (bit numbers):
3    - 0 -
4    |   |
5    5   1
6    |   |
7    - 6 -
8    |   |
9    4   2
10    |   |
11    - 3 -  .7
12 
13 // Display error: E r r \_
14 display.rawDigit(0, 0b01111001);
15 display.rawDigit(1, 0b01010000);
16 display.rawDigit(2, 0b01010000);
17 display.rawDigit(3, 0b00000000);
18 
19 // Display H character: \_ \_ \_ H
20 display.rawDigit(3, 0b01110110);
21 
22 // Display negative temperature: - 1 ` C
23 display.rawDigit(0, SEGMENTS\_MINUS);
24 display.digit(1, 1);
25 display.rawDigit(2, SEGMENTS\_DEGREE);
26 display.rawDigit(3, SEGMENTS\_CELSIUS);
27 
28 // Display rect
29 display.rawDigit(0, 0b00111001);
30 display.rawDigit(1, 0b00001001);
31 display.rawDigit(2, 0b00001001);
32 display.rawDigit(3, 0b00001111);
\end{DoxyCode}


\subsection*{Library dependencies}


\begin{DoxyItemize}
\item \href{https://github.com/Erriez/ErriezTM1637}{\tt Erriez T\+M1637} library
\end{DoxyItemize}

\subsection*{Library installation}

Please refer to the \href{https://github.com/Erriez/ErriezArduinoLibrariesAndSketches/wiki}{\tt Wiki} page.

\subsection*{Other Arduino Libraries and Sketches from Erriez}


\begin{DoxyItemize}
\item \href{https://github.com/Erriez/ErriezArduinoLibrariesAndSketches}{\tt Erriez Libraries and Sketches} 
\end{DoxyItemize}